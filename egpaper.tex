\documentclass[10pt,letterpaper,a4paper]{article}
\usepackage[UTF8]{ctex}
\usepackage{wacv}
\usepackage{times}
\usepackage{epsfig}
\usepackage{graphicx}
\usepackage{amsmath}
\usepackage{amssymb}
\usepackage{float} 
\usepackage{subfigure} 
\usepackage{xcolor}
\usepackage{academicons}
\usepackage{listings}
\usepackage[numbers, sort, comma, square]{natbib}
\usepackage[textwidth=18.2cm,textheight=23.2cm,paperheight=29cm,footskip=1cm]{geometry}

% 在hyperref之前包含其他包。

%%%%%%%%%%%%%%%%%%%%%%%%%%%%%%%%%%%%%%%%%%%%%%%%%%%%%%%%%%%%%%%%%%%%%%%%%%%%%%%%%%%%%%%%%%%%%%%%%%
%
%%% 重要-接下来的三行是至关重要的:
%               (1) 请输入您的本次提案类型{proposalType},OCT/OI/ORP/OE/OEG
%                   '****' 这是默认填充,你可以根据此来进行编辑您的提案类型.
%                       1.1 OCT,Openness Community 开放性研究
%                           提案的研究结果或过程等信息
%                       1.2 OI,Openness Idea 开放性参与
%                           参与某一提案的内容所提交的贡献
%                       1.3 ORP,Openness Request Proposal 请求提案
%                           提出一个想法,尽可能的描述
%                       1.4 OE,Openness Council 开放决策
%                           决定某一件事的信息,一般用于 OI 所进行发布的文件
%                       1.5 OEG,Openness Engineering Group 工程计划
%                           ORP 所完成的计划阶段
%               (2) 如果将取消 \openfile 注释
%                   没有取消注释将会同按照 ORP,Openness Request Proposal 类型上传
%                   取消了注释将会按照正式的 OE,Openness Council 类型文件上传
%                   两者的区别在于前者是需要社区讨论的提案,而后者是已经正式实施的提案
%               (3) OEG TEAM 分配的当前文件的页码
%                   该提案将会在社区期刊中进行发布,你需要根据 OEG TEAM 分配给你的页面进行替换
%                   前提是该文件已经是 openfile ,需要你取消注释。

%(1)
\def\proposalType{OEG} % **** 提案类型
\def\submissionOrg{Openness Engineering Group(OEG)} % **** 报送组织
\def\proposalId{255} % **** 提案编号
\def\planProject{45646} % **** 计划项目编号
\def\updates{\href{https://www.google.com}{2191}} % *** 更新提案
\def\issnNumber{2070-1721} % **** ISSN 编号,最终由 OEG 颁发,在 openfile 中使用,但一般情况下在 ORP

%(2)
%\openfile % *** 类型文件公开后请取消此注释

%(3)
\ifwacvfinal
\def\assignedStartPage{2022} % *** openfile 时的起始页,在第二阶段开始分配,第一阶段请忽略
\fi

% 如果你注释了hyperref,然后取消注释,你应该删除它
% egpaper。备用前重新运行latex (或者只是在第一层 latex上按下“q”
% 运行,让它结束,您应该清楚)
\ifwacvfinal
\usepackage[breaklinks=true,bookmarks=false]{hyperref}
\hypersetup{
   colorlinks=true,            %链接颜色
   urlcolor=blue,              %网址链接
   linkcolor=blue,             %锚(目标)文本的颜色
   anchorcolor=blue,           %内部链接颜色
   citecolor=blue              %参考文献引用颜色
   pdftitle={Overleaf Example},  %设置文档标题
   pdfpagemode=FullScreen,     %打开文件的方式
}
\else
\usepackage[pagebackref=true,breaklinks=true,colorlinks=true,bookmarks=false]{hyperref}
\hypersetup{
   colorlinks=true,            %链接颜色
   urlcolor=blue,              %网址链接
   linkcolor=blue,             %锚(目标)文本的颜色
   anchorcolor=blue,           %内部链接颜色
   citecolor=blue              %参考文献引用颜色
   pdftitle={Overleaf Example},  %设置文档标题
   pdfpagemode=FullScreen,     %打开文件的方式
}
\fi

%(4)
% 在 openfile 模式下是会由 OEG TEAM 分配的,但默认 ORP 的页码是从 1 开始的。
\ifwacvfinal
\setcounter{page}{\assignedStartPage}
\else
% 这个是当前 ORP 模式下的页码
\setcounter{page}{1}
\fi

\def\httilde{\mbox{\tt\raisebox{-.5ex}{\symbol{126}}}}

%%%%%%%%% 标题与副标题
\title{\LaTeX\ Author Guidelines for WACV Proceedings}
\def\subheading{你好世界} 
\begin{document}

\frontmatter
\maketitle
\newcommand{\orcid}[1]{\href{https://orcid.org/#1}{\textcolor[HTML]{blue}{\aiOrcid}}}

%\thispagestyle{empty}
%%%% 作者序列
% 1) 如果是单一作者,不要注明单一的合作地址。
% 2) 如果你需要两行或更多的author,使用\newauthor(参见下面的用法)
% 3) 所有作者都需要 (ORCID)[https://orcid.org/]
% 4) 短作者应该是A.U.索尔等或- A.U.索尔
\author[F. Author et al.]{%
First Author,$^{1}$\thanks{tjaa@tad.org.tr}\orcid{0000-0000-0000-0000}
A. N. Other,$^{2}$\orcid{0000-0000-0000-0000}
Third Author,$^{2,3}$\orcid{0000-0000-0000-0000}
\newauthor
Fourth Author$^{3}$\orcid{0000-0000-0000-0000}
\\
% 院校名单或作者单位、组织、团队等
% 1) 不要把 \\ 放在最后一段中
$^{1}$University, Department, City Post Code, Country\\
$^{2}$Department, Institution, Street Address, City Postal Code, Country\\
$^{3}$Another Department, Different Institution, Street Address, City Postal Code, Country%
\vspace*{12pt}}

%%%%%%%%% 抽象摘要
\begin{abstract}
   摘要要在顶部用斜体显示  
   左边一栏,作者和所属机构以下  
   信息。 使用“摘要”这个词作为标题,在12点  
   时间,黑体字,初始时相对于列居中  
   大写。 摘要应采用10点单行距格式。  
   在摘要之后留下两行空白,然后开始正文。  
   看看以前的WACV摘要,了解一下风格和长度。  
\end{abstract}


%%%%%%%%%%%%%%%%%%%%%%%%%%%%%%%%%%%
% 自动生成的内容表,数字列表和表列表 %   
%%%%%%%%%%%%%%%%%%%%%%%%%%%%%%%%%%%
\newpage\tableofcontents  % 目录
% \listoffigures  % 插图目录
% \listoftables   % 表格目录

%%%%%%%%% 主要内容
\twocolumn % 分栏
% 对于基础的 LeTex 语法,欢迎阅读 https://en.wikibooks.org/wiki/LaTeX/Document_Structure#Sectioning_commands

\begin{sloppypar}
\section{介绍}
\\
请按照下面列出的步骤提交您的手稿  
IEEE 计算机协会出版社以及 WACV 。这个
风格指南和风格现在有几个重要的修改(例
如,增加了更加详细的左右信息和文件作用),
所以所有作者  应该读一下这个新版本。  

%-------------------------------------------------------------------------
\subsection{关于语言}
所有手稿必须是英文或中文简体、繁体的,当然
我们也支持你在 \LaTeX 中使用相关的宏包,只
要最后的效果能正常编译即可。

对于编译,如果你对 \LaTeX 了解的并不深,建
议在编写完 .tex 文件后,直接用\verb'latex
mk -pdf <file.tex>' 的形式进行编译(前提
是你已经进入到了当前的文件夹目录内)

如果未使用 \href{https://www.overleaf.com/}{overleaf} 类似的这种云 \LaTeX
环境,那么本地上请确保拥有了 \href{https://www.tug.org/texlive/}{Tex Live} 了本地的环境,
且具有了环境变量,可以通过命令诸如 \verb'latexmk' 这种命令进行编译。

当然你也可以通过 \verb'TeXworks editor' 或是 \verb'TeXstudio' 等主流针对 \LaTeX 的编辑工具
进行写作和编译。


\subsection{代码块}
通常在 \LaTeX 中使用代码块可以通过 listings 
和 xcolor 宏包进行,前者用于支持代码块,后者
支持其代码高亮。
\lstset{
  language=Matlab,                                %代码语言使用的是matlab
  keywordstyle=\color{blue!90}\bfseries,          %代码关键字的颜色为蓝色,粗体
  commentstyle=\color{red!10!green!70}\textit,    % 设置代码注释的颜色
  showstringspaces=false,                         %不显示代码字符串中间的空格标记
  numbers=left,                                   % 显示行号
  numberstyle=\footnotesize,                      % 行号字体
  stringstyle=\ttfamily,                          % 代码字符串的特殊格式
  breaklines=true,                                %对过长的代码自动换行
  extendedchars=false,                            %解决代码跨页时,章节标题,页眉等汉字不显示的问题
% escapebegin=\begin{CJK*},escapeend=\end{CJK*},  % 代码中出现中文必须加上,否则报错
  texcl=true
}
\begin{lstlisting}
   clear;
   clc;
   % 此为Paper中Scetion5代码
   % 拉普拉斯矩阵如下
   L = [2, -1, -1;
        0,  1, -1;
       -1,  0,  1;];
   % Equation 18
   syms t;
   R = expm(-L * t);
   R1 = limit(R, t, inf);
   R2 = eigenVector * lefteigenVector'; 
\end{lstlisting}

对于上述的代码块,可以在 \verb'\begin{lstlisting}' 和 
\verb'\end{lstlisting}' 之间进行更改或添加 Code 内,
或者使用其他方式。


\end{lstlisting}'
\subsection{纸的长度}
对于提案,{\bf开放社区和联合现在均不收任何形式的版面费用,}
对此文章的长度也页面也不会受到任何限制,在整个联合现在提供的
服务内,{\bf均不收任何形式的额外费用}。  

\\任何形式的提案,都会经过二次审核和编辑,但是在该提案一开始提
交时,是直接呈现的,之后的任何修改都会在第一次发布的基础上进
行更新。

%-------------------------------------------------------------------------
\subsection{标尺}
\LaTeX\ 样式定义了一个打印的标尺,它应该出现在
提交审核的版本。 提供标尺是为了
审稿人可以对提案中的特定行发表评论,而无需
迂回曲折。 如果您正在使用非 \LaTeX\
文件准备系统,请安排一个等效的尺子
出现在最终输出页面上。

尺子的存在与否
不应更改页面上任何其他内容的外观。 这
相机准备好的副本不应包含标尺。 (\LaTeX\ 用户可以取消注释
文档序言中的\verb'\openfile'命令。)\\审稿人:
请注意,标尺测量值与纸张中的线条不对齐
--- 事实证明,当论文包含
许多数字和方程式,完成后看起来很难看。 只需使用小数
参考文献(例如\这一行是 $087.5$),尽管在大多数情况下会
预计大致位置就足够了。

对于标尺,他只出现在开放社区提案中,以及大会
(Openness Council) 的讨论,在此之后的发布将
会通过此命令取消标尺和页头信息。
%-------------------------------------------------------------------------

\subsection{页头信息}
\verb'\submissionOrg' 是提案需要发送的组织机构,通常是联合现在的相
关体系结构团队。\verb'\proposalId' 是当前提案的编号,可以留空等待 OEG
进行分配。

\verb'\proposalType' 是当前提案的类型,一般填缩写,可以参考 .tex 文件的
开头注释。\verb'\updates' 当前提案所更新、跟进的提案内容,直接应用其提案
编号即可。

\verb'\planProject' 为某个项目提交提案,那么该提案将会默认归类到该项目提
案内。

\verb'\issnNumber' 即当前 ISSN 的编码,通常由OEG在大会发布后进行颁发
但通常在会议结束后通过期刊统一发布。

\subsection{数学}
请给你所有的部分和显示的方程式编号。 这是
重要的是读者能够参考任何特定的方程式。 只是
因为你没有在文中提到它并不意味着未来的读者
可能不需要参考它。 必须使用很麻烦
像“第 3 页顶部第二个等式”这样的迂回说法
1''。 (请注意,尺子不会出现在最终副本中,因此
等式编号的替代方案)。 所有作者都将从阅读中受益
Mermin 对如何写数学的描述。

\subsection{盲审}
许多作者误解了盲人匿名化的概念
审查。 盲审并不意味着必须删除
对自己作品的引用——事实上,通常不可能
审阅一篇论文,除非之前的引用是已知的,并且
可用的。

盲审意味着您不使用“我的”或“我们的”这些词
引用以前的工作时。 就这些。 (但见下文
技术报告。)

说“这建立在露西·史密斯 [1] 的工作之上”并不是说
你是露西·史密斯; 它说你在她的基础上
工作。 如果你是史密斯和琼斯,不要说“正如我们在
[7]'',说“正如史密斯和琼斯在 [7] 中展示的那样”,并在结尾处
论文,包括参考文献 7,就像任何其他引用的作品一样。

只要求被拒绝的糟糕论文的示例:
\begin{quote}
\begin{center}
   对 frobnicable foo 过滤器的分析。
\end{center}

在本文中,我们对我们的性能进行了分析
    以前的论文 [1],并表明它不如所有
    以前已知的方法。 为什么上一篇论文是
    接受没有这种分析是超出我的。

    [1] 删除以进行盲审
\end{quote}


可接受论文的示例:

\begin{quote}
\begin{center}
   对 frobnicable foo 过滤器的分析。
\end{center}

在本文中,我们将介绍性能分析
    Smith \ etal [1] 的论文,并证明它不如
    所有以前已知的方法。 为什么上一篇论文
    没有这个分析就被接受了,这超出了我的理解。

    [1] Smith, L and Jones, C. ``The frobnicatable foo
    filter, a fundamental contribution to human knowledge''.
    Nature 381(12), 1-213.
\end{quote}

如果您同时提交给另一个会议,
涵盖类似或重叠的材料,您可能需要参考
提交以解释差异,就像您将
之前发表过相关作品。 在这种情况下,包括
匿名并行提交~\cite{Authors20} 作为附加材料和
引用它作为
\begin{quote}
[1] Authors. ``The frobnicatable foo filter'', F\&G 2020 Submission ID 324,
作为附加材料提供
\end{quote}

最后,你可能觉得需要告诉读者更多细节可以
在其他地方找到,并将其提交给技术报告。 会议用
提交,论文必须独立,而不是 {\em 要求}
审阅者前往技术报告了解更多详情。 因此,你可以说
论文正文``可以找到更多细节
在~\cite{Authors20b}''。 然后提交技术报告作为附加材料。
同样,您可能不会认为审稿人会阅读此材料。

有时你的论文是关于你使用工具测试的问题
众所周知,它仅限于单一机构。 例如,
假设是 1969 年,你解决了阿波罗着陆器上的一个关键问题,
并且您认为 WACV 70 的观众希望了解您的
解决方案。 这项工作是您 1968 年著名论文的发展,题为
``零-g frobnication:如何成为世界上唯一可以访问的人
阿波罗着陆器源代码让我们在派对上惊叹不已,作者 Zeus \etal。

您可以像处理其他任何文件一样处理此文件。 不要写“我们展示如何
改进我们以前的工作 [Anonymous, 1968]。 这次我们测试了
月球着陆器上的算法[着陆器名称已删除以进行盲审]''。
那将是愚蠢的,并且会立即确定作者。 反而
写下:
\begin{quotation}
\noindent
我们描述了一个零重力摩擦系统。 这
    system 是新的,因为它处理以下情况:
    A, B. 以前的系统 [Zeus et al. 1968]没有
    妥善处理案例 B。 我们的处理方法包括
    bar 积分中的 foo 项。

    ...

    拟议的系统与阿波罗集成
    登月器,一路去月球,不要
    你知道。 它显示了以下行为
    这表明我们解决案例 A 和 B 的效果如何: ...
\end{quotation}
如您所见,上面的文字遵循标准的科学惯例,
读起来比第一个版本好,并且没有明确地将您命名为
作者。 审稿人可能认为新论文很可能是
由 Zeus \etal 编写,但无法根据该猜测做出任何决定。
他或她必须确保没有其他作者可以
签约解决问题 B.
\medskip

\noindent
FAQ\medskip\\
{\bf Q:} 确认可以吗?\\
{\bf A:} 不,将它们留作最终副本。\medskip\\
{\bf Q:} 我如何引用我在公开挑战中报告的结果?
{\bf A:} 为了符合双盲评审政策,您可以在论文中报告
其他挑战参与者的结果以及您的结果。 但是,对于您的
结果,您不应该表明自己的身份,也不应该提及您参与了
挑战。 而是参考您论文中提出的方法提出您的结果,并
根据与其他结果的实验比较得出结论。\medskip\\
\begin{figure}[t]
\begin{center}
\fbox{\rule{0pt}{2in} \rule{0.9\linewidth}{0pt}}
   %\includegraphics[width=0.8\linewidth]{egfigure.eps}
\end{center}
   \caption{Example of caption.  It is set in Roman so that mathematics
   (always set in Roman: $B \sin A = A \sin B$) may be included without an
   ugly clash.}
\label{fig:long}
\label{fig:onecol}
\end{figure}

\subsection{各种各样的}
\noindent
比较以下内容:\\
\begin{tabular}{ll}
 \verb'$conf_a$' &  $conf_a$ \\
 \verb'$\mathit{conf}_a$' & $\mathit{conf}_a$
\end{tabular}\\
参见 The \TeX book, p165。

\eg 之后的空格,意思是“例如”,不应该是
句尾空格。 所以 \eg 是正确的, {\em e.g.} 不是。 提供的
\verb'\eg' 宏负责这一点。

引用多位作者的论文时,您可以使用 ``et alia'' 来节省空间,
缩写为 ``\etal'' (不是 ``{\em et.\ al.}'' 因为 ``{\em et}'' 是一个完整的词。)
但是,仅当有三个或更多作者时才使用它。 就这样
以下是正确的:``
    Frobnication最近很流行。
    它由 Alpher~\cite{Alpher02} 引入,随后由
    Alpher 和 Fotheringham-Smythe~\cite{Alpher03},以及 Alpher \etal~\cite{Alpher04}。''

这是不正确的:``... 随后由 Alpher \etal~\cite{Alpher03} 开发...''
因为reference~\cite{Alpher03} 只有两个作者。 如果您使用
提供了 \verb'\etal' 宏,那么你不必担心双句号
用在句子的结尾时,如在 Alpher \etal 中。

对于这种引文风格,以数字形式保留多个引文(不是
按时间顺序),所以更喜欢 \cite{Alpher03,Alpher02,Authors20}
\引用{Alpher02,Alpher03,Authors20}。


\begin{figure*}
\begin{center}
\fbox{\rule{0pt}{2in} \rule{.9\linewidth}{0pt}}
\end{center}
   \caption{Example of a short caption, which should be centered.}
\label{fig:short}
\end{figure*}

%------------------------------------------------------------------------
\section{格式化你的论文}
所有文本必须采用两列格式。 总允许宽度
文本区域是 $6\frac78$ 英寸(17.5 厘米)宽$8\frac78$ 英寸(22.54
厘米)高。 列是$3\frac14$ 英寸(8.25 厘米)宽,带有
$\frac{5}{16}$ 英寸(0.8 厘米)的空间。 主标题(在
第一页)应从距页面顶部边缘 1.0 英寸(2.54 厘米)处开始
页。 第二页及后续页面应从距 1.0 英寸(2.54 厘米)开始
顶部边缘。 在所有页面上,下边距应为 1-1/8 英寸(2.86
厘米)从页面的底部边缘 $8.5 \times 11$-英寸纸; 适用于 A4
纸,距离纸张底部边缘约 1-5/8 英寸(4.13 厘米)
页。

%-------------------------------------------------------------------------
\subsection{页边距和页码}

必须保留所有印刷材料,包括文本、插图和图表
在 6-7/8 英寸(17.5 厘米)宽 x 8-7/8 英寸(22.54 厘米)的打印区域内
高的。

页码应与页码放在页脚中,居中且 0.75
距离页面底部英寸,并使其从正确的页面开始
数字而不是示例中的 9876。 为此,请找到 secounter
行(在此文件中的第 33 行附近)并将页码更新为
\begin{verbatim}
\setcounter{page}{123}
\end{verbatim}
其中数字 123 是您分配的起始页。

%-------------------------------------------------------------------------
\subsection{类型样式和字体}
在指定 Times 的地方,也可以使用 Times Roman。如果两者都不是
在您的文字处理器上可用,请使用最接近的字体
出现在您有权访问的时代。

主题。将标题居中距顶部边缘 1-3/8 英寸(3.49 厘米)
第一页。标题应为 Times 14 点黑体字。
名词、代词、动词、形容词的首字母大写
副词;不要大写冠词、并列连词或
介词(除非标题以这样的词开头)。留两个空白
标题后的行。

AUTHOR NAME(s) 和 AFFILIATION(s) 将在标题下方居中
并以 Times 12 点非粗体字印刷。该信息是
后跟两个空行。

ABSTRACT 和 MAIN TEXT 将采用两列格式。

正文。在 10 点 Times 中键入正文,单行距。不使用
双间距。所有段落应缩进 1 pica(约 1/6
英寸或 0.422 厘米)。确保你的文字是完全合理的——也就是说,
向左冲洗和向右冲洗。请不要放置任何额外的空白
段落之间的线条。

图和表格标题应为 9 点罗马字体,如
图~\ref{fig:onecol} 和~\ref{fig:short}。短标题应居中。

\noindent 标注应该是 9 点 Helvetica,非粗体类型。
最初只大写章节标题的第一个单词和第一个,
二级和三级标题。

一阶标题。 (例如,{\large \bf 1. Introduction})
应该是 Times 12 号粗体,首字母大写,左对齐,
前一空行,后一空行。

二阶标题。 (例如,{ \bf 1.1. 数据库元素})
应该是 Times 11 号黑体,首字母大写,左对齐,
前一空行,后一空行。如果您需要三阶
标题(我们不鼓励),最初使用 10 点 Times,粗体字
大写,左对齐,前面有一个空行,后面是一个句点
和你的文字在同一行。

%-------------------------------------------------------------------------
\subsection{脚注}
请使用脚注\footnote {这是脚注的样子。 它
经常会分散读者对论点的注意力。} 谨慎。
确实,尽量避免使用脚注,并包括必要的外围设备
中的观察
文本(在括号内,如果您愿意,如本句所示)。 如果你
希望使用脚注,请将其放在页面上的列底部
它被引用。 使用 Times 8 点类型,单行距。

%-------------------------------------------------------------------------
\subsection{参考}
列出并编号 9 点时代的所有参考书目,
单行距,在论文的末尾。 文中提及时,
将引文编号括在方括号中,对于
示例~\citet{Authors20}。 在适当的情况下,包括
参考书籍的编辑。

\begin{table}
\begin{center}
\begin{tabular}{|l|c|}
\hline
Method & Frobnability \\
\hline\hline
Theirs & Frumpy \\
Yours & Frobbly \\
Ours & Makes one's heart Frob\\
\hline
\end{tabular}
\end{center}
\caption{Results.   Ours is better.}
\end{table}

%-------------------------------------------------------------------------
\subsection{插图、图表和照片}
所有图形都应居中。 请确保您希望的任何点
make 可以在纸张的打印副本中解析。 调整图形中的字体大小
匹配正文中的字体,并选择渲染的线宽
有效地印刷。 许多读者(和评论者),甚至是电子版的
副本,将选择打印您的论文以便阅读。 你不能
坚持他们不这样做,因此不能假设他们可以
放大以查看图形上的微小细节。

在 \LaTeX 中放置图形时,几乎总是最好使用
includegraphics,并将图形宽度指定为的倍数
线宽如下例所示
{\small\begin{verbatim}
   \usepackage[dvips]{graphicx} ...
   \includegraphics[width=0.8\linewidth]
                   {myfile.eps}
\end{verbatim}
}


%-------------------------------------------------------------------------
\subsection{颜色}
请参考 WACV 2023 网页上的作者指南有关在文档中使用颜色的讨论。(\url{http://wacv2023.thecvf.com/submission/})

%------------------------------------------------------------------------
\section{最终副本}
提交时必须附上已签署的 IEEE 版权发布表
你完成的论文。 我们必须在您的论文被提交之前拥有此表格
在议事录中公布。

如有任何问题,请咨询负责这些的制作编辑
IEEE 计算机学会出版社的会议记录:
\url{https://www.computer.org/about/contact}。


{\small
\bibliographystyle{unsrtnat}
\bibliography{egbib}
}
\begin{sloppypar}
\end{document}
